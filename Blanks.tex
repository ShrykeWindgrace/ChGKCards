\documentclass[a4paper, 12pt]{article}
% \usepackage[utf8]{inputenc}
\usepackage{fontspec}
\usepackage[russian]{babel}
\usepackage{graphicx}
\usepackage{geometry}
\usepackage{fp}
\usepackage{forloop}
% \usepackage{etoolbox}
\geometry{
	total={210mm,297mm},
	left=0mm,
	% right=0mm,
	% top=0mm,
	bottom=0mm,
}
\topskip0pt

\newcommand{\blNew}[2]{\noindent\fbox{\begin{minipage}[t][71.5mm][t]{50mm}\rotatebox[origin=rb]{90}{Вопрос:~#2}\vfill\rotatebox{90}{Команда:~#1}\end{minipage}}}%

\newcommand{\totalteamsPO}{4}% total number of questions plus one
\newcommand{\totalquestsPO}{37}% total number of questions plus one

\begin{document}\setmainfont{Lucida Console}%
\newcounter{teams}%number of teams
\newcounter{quest}%questions
% \newcounter{straw}%%number of additional blanks; printed without question numbers
%NB! some printers like to add margins, so the final view will not be exactly the same as in preview
\noindent\forloop{teams}{1}{\value{teams}<\totalteamsPO}{\noindent\forloop{quest}{1}{\value{quest}<\totalquestsPO}{\blNew{\theteams}{\thequest}\FPeval{\res}{(\theteams-1)*(\totalquestsPO-1)+\thequest}\FPeval{\resu}{trunc(\res-(4*trunc(\res/4,0)),0)}\FPifeq{\resu}{0}{\newline}\else{}\fi}}%
\end{document}