\documentclass[a4paper, 12pt, landscape]{article}
% \usepackage[utf8]{inputenc}
\usepackage{fontspec}
\usepackage[russian]{babel}
\usepackage{graphicx}
\usepackage{geometry}
\usepackage[nomessages]{fp}
\usepackage{forloop}
\usepackage{etoolbox}
\usepackage{xcolor}
% \usepackage{background}
% \backgroundsetup{contents=Confidential,color=blue}
% \backgroundsetup{color=blue}
\geometry{
	total={297mm,210mm},
	left=0mm,
	% right=0mm,
	% top=0mm,
	bottom=0mm,
}
\topskip0pt
\newcommand{\imago}{\[\textcolor{lightgray}{\hat{(}\grave{\odot}{\scriptstyle\vee}\acute{\odot}\hat{)}}\]}

\newcommand{\blNew}[2]{\noindent\fbox{\begin{minipage}[t][50mm][t]{71.75mm}{Вопрос:~#2}\hfill{Команда:~#1}\vfill\imago\vfill\hspace{0pt}\end{minipage}}}%

\newcommand{\totalteamsPO}{7}% total number of teams +1
% \newcommand{\totalteamsPR}{3}% total number of questions
\newcommand{\totalquestsPO}{7}% total number of questions+1
% \newcommand{\totalquestsPR}{36}% total number of questions
% \newcommand{\totalreservePO}{4}% total number of reserve blanks per team

\newtoggle{printreserve}
\togglefalse{printreserve}
% \toggletrue{printreserve} %uncomment this line if you want to print reserve blanks for teams
\newtoggle{printreserveTeamNumbers} %has no effect if printreserve==false. 
\togglefalse{printreserveTeamNumbers}
% \toggletrue{printreserveTeamNumbers} %uncomment this line if you want to print team numbers on reserve blanks


\begin{document}\setmainfont{Lucida Console}%
% \FPupn{totalteamsPOne}{8 1 +}%
% \FPeval{\totalquestsPO}{\totalquests+1}%
\newcounter{teams}%number of teams
\newcounter{quest}%questions
\newcounter{reserve}%number of additional blanks; printed without question numbers
\newcounter{stepTZ}
\setcounter{stepTZ}{0}
%NB! some printers like to add margins, so the final view will not be exactly the same as in preview
\noindent\forloop{teams}{1}{\value{teams}<\totalteamsPO}{\noindent\forloop{quest}{1}{\value{quest}<\totalquestsPO}{\blNew{\theteams}{\thequest}\stepcounter{stepTZ}\FPeval{\resu}{trunc(\thestepTZ-(4*trunc(\thestepTZ/4,0)),0)}\FPifeq{\resu}{0}{\newline}\else{}\fi}}\iftoggle{printreserve}{\newline\noindent\forloop{teams}{1}{\value{teams}<\totalteamsPO}{\forloop{reserve}{0}{\value{reserve}<4}{\blNew{\iftoggle{printreserveTeamNumbers}{\theteams}{}}{}}\newline}}{}
\end{document}